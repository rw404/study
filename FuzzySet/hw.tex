\documentclass[12pt]{article}
\usepackage[utf8]{inputenc} % кодировка utf8
\usepackage[english,russian]{babel}
\usepackage{mathtext} % кириллица в формулах \text{}
\usepackage{amsmath} % пробелы между слов в \text{}
\usepackage{indentfirst} % красная строка
\usepackage{graphicx} % отображение картинок
\usepackage{subfigure} % отображение коллажа картинок
\usepackage{amssymb} % отображение математической нотации
\usepackage{multirow} % для разделения строк в таблицах
\usepackage{hhline} % для таблиц \hhline{...}
\usepackage{array} % для таблиц m{...}
\usepackage[table]{xcolor} % раскраска ячеек таблицы
% \usepackage[nottoc,notlot,notlof]{tocbibind} % для отображения списка литературы в содержании
\usepackage{setspace} % интервал между строк
\usepackage[
    left=30mm, right=20mm,
    top=20mm, bottom=20mm,
    bindingoffset=0mm
    ]{geometry} % настройка границ документа
\linespread{1.5}
\usepackage{fancyhdr}
\usepackage{ccaption}
\usepackage{hyperref} % гиперссылки по тексту
\usepackage{amsfonts} % \mathbb{...}
\usepackage{algorithm2e}
\usepackage{amsthm}

\hypersetup{
	colorlinks=true,
	linkcolor=blue,
	filecolor=magenta,
	urlcolor=cyan,
}

\addto\captionsrussian{\def\refname{Литература}}



\begin{document}

\section{Теория}
\subsection{Треугольная норма(t-норма)}

Треугольная норма(t-норма) $T:[0, 1]\times[0, 1]\rightarrow[0, 1]$, удовлетворяющая условиям:

\begin{itemize}
    \item $T(0, 0) = 0, T(\mu_a, 1) = \mu_a$ (ограниченность);
    \item $T(\mu_a, \mu_b) = T(\mu_b, \mu_a)$ (коммутативность);
    \item $T(\mu_a, T(\mu_b, \mu_c)) = T(T(\mu_a, \mu_b), \mu_c)$ (ассоциативность);
    \item $T(\mu_a, \mu_b)\le T(\mu_c, \mu_d)$, если $\mu_a\le\mu_c$ и $\mu_b\le\mu_d$ (монотонность).
\end{itemize}

\subsubsection{Пример}
\begin{itemize}
    \item $T_m(\mu_a, \mu_b) = min(\mu_a, \mu_b)$;
    \item $T_p(\mu_a, \mu_b) = \mu_a*\mu_b$;
    \item $T_p(\mu_a, \mu_b) = \mu_a*\mu_b$;
    \item $T_d(\mu_a, \mu_b) =
\begin{cases}
\mu_a &, \text{если }\mu_b=1\\
\mu_b &, \text{если }\mu_a=1\\
0 &,  \text{в других случаях}\\
\end{cases}$
\end{itemize}

\subsection{Треугольная конорма(t-конорма)}

Треугольная конорма(t-конорма) $\perp:[0, 1]\times[0, 1]\rightarrow[0, 1]$, удовлетворяющая условиям:

\begin{itemize}
    \item $\perp(1, 1) = 1, \perp(\mu_a, 0) = \mu_a$ (ограниченность);
    \item $\perp(\mu_a, \mu_b) = \perp(\mu_b, \mu_a)$ (коммутативность);
    \item $\perp(\mu_a, \perp(\mu_b, \mu_c)) = \perp(\perp(\mu_a, \mu_b), \mu_c)$ (ассоциативность);
    \item $\perp(\mu_a, \mu_b)\le \perp(\mu_c, \mu_d)$, если $\mu_a\le\mu_c$ и $\mu_b\le\mu_d$ (монотонность).
\end{itemize}

\subsubsection{Пример}
\begin{itemize}
    \item $\perp_m(\mu_a, \mu_b) = max(\mu_a, \mu_b)$;
    \item $\perp_p(\mu_a, \mu_b) = \mu_a+\mu_b-\mu_a*\mu_b$;
    \item $\perp_l(\mu_a, \mu_b) = min( \mu_a+\mu_b, 1)$;
    \item $\perp_d(\mu_a, \mu_b) =
\begin{cases}
\mu_a &, \text{если }\mu_b=0\\
\mu_b &, \text{если }\mu_a=0\\
1 &,  \text{в других случаях}\\
\end{cases}$  
\end{itemize}

\subsection{Специальные t-нормы и t-конормы}

\subsubsection{Майор-Торренс}

$$
    T(\mu_a, \mu_b) =
\begin{cases}
max(\mu_a+\mu_b-\lambda, 0) &, \text{ если }\lambda\in[0, 1] \text{ и }(\mu_a, \mu_b)\in[0,\lambda]^2\\
min(\mu_a, \mu_b) &, \text{ если } \lambda=0 \text{ или }\mu_a>\lambda \text{ или } \mu_b>\lambda\\
\end{cases}
$$

$$
\perp(\mu_a, \mu_b) =
\begin{cases}
min(\mu_a+\mu_b+\lambda-1, 1) &, \text{ если }\lambda\in[0, 1] \text{ и }(\mu_a, \mu_b)\in[1-\lambda, 1]^2\\
max(\mu_a, \mu_b) &, \text{ если } \lambda=0 \text{ или }\mu_a<1-\lambda \text{ или } \mu_b<1-\lambda\\
\end{cases} 
$$

\subsubsection{Ягер}

$$
    T(\mu_a, \mu_b) =
	\begin{cases}
    	max\left(1-\left((1-\mu_a)^\lambda+(1-\mu_b)^\lambda\right)^{\frac{1}{\lambda}}, 0\right) &, \text{ если }\lambda\in(0, +\infty)\\
    	T_d(\mu_a, \mu_b) &, \text{ если }\lambda=0\\
    	T_m(\mu_a, \mu_b) &, \text{ если }\lambda=\infty\\
	\end{cases}
$$
$$
\perp(\mu_a, \mu_b) =
\begin{cases}
    min\left((\mu_a^\lambda+\mu_b^\lambda)^{\frac{1}{\lambda}}, 1\right) &, \text{ если }\lambda\in(0, +\infty)\\
    \perp_d(\mu_a, \mu_b) &, \text{ если }\lambda=0\\
    \perp_m(\mu_a, \mu_b) &, \text{ если }\lambda=\infty\\
\end{cases}
$$

\section{Домашнее задание}

\begin{enumerate}
    \item Доказать: $T_d\le T_l\le T_p\le T_m$;
    \item Доказать: $\forall t\text{-нормы } T: T_d\le T \le T_m$;
    \item Доказать: $\perp_d\ge \perp_l\ge \perp_p\ge \perp_m$;
    \item Доказать: $\forall t\text{-конормы } \perp: \perp_d\le \perp \le \perp_m$;
    \item Проверить выполнение свойств t-норм и t-конорм для t-нормы и t-конормы Майора-Торренса;
    \item Проверить выполнение свойств t-норм и t-конорм для t-нормы и t-конормы Ягера.
\end{enumerate}

\subsection{Доказать: $\forall t\text{-нормы } T: T_d\le T \le T_m$}

\begin{proof}
    \begin{enumerate}
        \item Из свойств $T(\mu_a, \mu_b) \le T(\mu_c, \mu_d), \text{ если } \mu_a\le\mu_c, \mu_b\le\mu_d$, $T(0, 0) = 0, T(\mu_a, 1) = \mu_a$ => $\forall T \forall (\mu_a, \mu_b)\in[0,1]^2$:
            \begin{itemize}
                \item $T(\mu_a, \mu_b) \le T(\mu_a, 1)=\mu_a$;
                \item $T(\mu_a, \mu_b) \le T(1, \mu_b)=\mu_b$.
            \end{itemize}
        \item На границе $[0, 1]^2$ $T(0, 0)=T(0, 1)=T(1, 0)=0, T(1, 1)=1$;
        \item $\forall(\mu_a, \mu_b)\in[0, 1]^2$ и $\forall T$
    $T(\mu_a, \mu_b)\ge 0=T_d(\mu_a, \mu_b)$ и $T(\mu_a, \mu_b)\le min(\mu_a, \mu_b)$(п.1).
    \end{enumerate}
\end{proof}

\subsection{Доказать: $T_d\le T_l\le T_p\le T_m$}

\begin{proof}
    \begin{enumerate}
        \item Докажем, что $max(0, \mu_a+\mu_b-1)\le\mu_a*\mu_b$:
            \begin{enumerate}
                \item 1-ый случай, когда $\mu_a+\mu_b-1\le0\le\mu_a*\mu_b$;
                \item 2-ой случай, когда $\mu_a+\mu_b-1>0 => \mu_a+\mu_b-1-\mu_a*\mu_b=-(1-\mu_a)*(1-\mu_b)\le 0$
            \end{enumerate}
        \item Получили, что $T_l\le T_p$;
        \item Тогда используем утверждение из 2-ой задачи и получаем, что 
   $T_d\le T_l\le T_p\le T_m$
    \end{enumerate}
\end{proof}

\subsection{Доказать: $\forall t\text{-конормы } \perp: \perp_d\le \perp \le \perp_m$}

\begin{proof}
    \begin{enumerate}
        \item Из свойств конормы $\perp(\mu_a, \mu_b)\le\perp(\mu_c, \mu_d) <= \mu_a\le\mu_c, \mu_b\le\mu_d$
   и $\perp(1, 1)=1, \perp(\mu_a, 0)=\mu_a$ получаем
   $\forall \perp \forall (\mu_a, \mu_b)\in[0, 1]^2$:
            \begin{itemize}
                \item $\perp(\mu_a, \mu_b) \ge \perp(\mu_a, 0)=\mu_a$;
                \item $\perp(\mu_a, \mu_b) \ge \perp(0, \mu_b)=\mu_b$.
            \end{itemize}
        \item На границе $[0, 1]^2 \perp(0, 0)=0, \perp(0, 1)=\perp(1, 0)=\perp(1, 1)=1$;
        \item $\forall(\mu_a, \mu_b)\in[0, 1]^2$ и $\forall \perp: \perp(\mu_a, \mu_b)\le 1=\perp_d(\mu_a, \mu_b)$ 
   и $\perp(\mu_a, \mu_b)\ge max(\mu_a, \mu_b)=\perp_m(\mu_a, \mu_b)$.
    \end{enumerate}
\end{proof}

\subsection{Доказать: $\perp_d\ge \perp_l\ge \perp_p\ge \perp_m$}

\begin{proof}
    \begin{enumerate}
        \item Докажем, что $min(\mu_a+\mu_b, 1)\ge\mu_a+\mu_b-\mu_a*\mu_b$:
            \begin{enumerate}
                \item 1-ый случай, когда 
      $\mu_a+\mu_b\ge1 => |1-\mu_a-\mu_b+\mu_a*\mu_b\ge0| => 1\ge \mu_a+\mu_b-\mu_a*\mu_b$; 
                \item 2-ой случай, когда $\mu_a+\mu_b<1 => \mu_a+\mu_b-\mu_a-\mu_b+\mu_a*\mu_b \ge 0$;
            \end{enumerate}
        \item Тогда используя решение п.4
   $\perp_d\ge\perp_l\ge\perp_p\ge\perp_m$.
    \end{enumerate}
\end{proof}

\subsection{Проверить выполнение свойств t-норм и t-конорм для t-нормы и t-конормы
Майора-Торренса}

\subsubsection{t-норма}

$$
T(\mu_a, \mu_b) =
\begin{cases}
max(\mu_a+\mu_b-\lambda, 0) &, \text{ если }\lambda\in[0, 1] \text{ и }(\mu_a, \mu_b)\in[0,\lambda]^2\\
min(\mu_a, \mu_b) &, \text{ если } \lambda=0 \text{ или }\mu_a>\lambda \text{ или } \mu_b>\lambda\\
\end{cases}
$$

\par Проверка свойств:
\begin{enumerate}
    \item $$ 
      T(0, 0) =
      \begin{cases}
         max(-\lambda, 0) &, \text{ если }\lambda\in[0, 1] \text{ и }(\mu_a, \mu_b)\in[0,\lambda]^2\\
         min(0, 0) &, \text{ если } \lambda=0 \text{ или }\mu_a>\lambda \text{ или } \mu_b>\lambda\\ 
      \end{cases} = 0;
      $$
      $$
      T(\mu_a, 1) = min(\mu_a, 1) = \mu_a;
      $$
  \item Комутативность очевидна;
  \item $$ 
      T(\mu_a, T(\mu_b, \mu_c)) = 
      \begin{cases}
         max(\mu_a+max(\mu_b+\mu_c-\lambda, 0)-\lambda, 0)\\
         min(\mu_a, min(\mu_b, \mu_c)) \\ 
      \end{cases} = 
  	$$
  	$$
      \begin{cases}
         max(max(\mu_a+\mu_b-\lambda, 0)+\mu_c-\lambda, 0) \\
         min(min(\mu_a, \mu_b), \mu_c) \\ 
      \end{cases}
      $$
  \item Монотонность следует из монотонности максимума и минимума.
\end{enumerate}

\subsubsection{t-конорма}

$$
   \perp(\mu_a, \mu_b) =
   \begin{cases}
      min(\mu_a+\mu_b+\lambda-1, 1) &, \text{ если }\lambda\in[0, 1] \text{ и }(\mu_a, \mu_b)\in[1-\lambda, 1]^2\\
      max(\mu_a, \mu_b) &, \text{ если } \lambda=0 \text{ или }\mu_a<1-\lambda \text{ или } \mu_b<1-\lambda\\
   \end{cases}
$$

Проверка свойств:
\begin{enumerate}
    \item $$
      \perp(1, 1) = 
      \begin{cases}
         min(1+\lambda, 1) \\
         max(1, 1) \\ 
      \end{cases} = 1;
      $$
      $$ 
      \perp(\mu_a, 0) = 
      \begin{cases}
         min(\mu_a+\lambda-1, 1) \\
         max(\mu_a, 0) \\ 
      \end{cases} = \mu_a
      $$
  \item Коммутативность и монотонность следует из монотонности максимума и минимума;
  \item Ассоциативность, как и в случае t-нормы.
\end{enumerate}

\subsection{Проверить выполнение свойств t-норм и t-конорм для t-нормы и t-конормы Ягера}

\subsubsection{t-норма}

$$
   T(\mu_a, \mu_b) =
   \begin{cases}
      max\left(1-\left((1-\mu_a)^\lambda+(1-\mu_b)^\lambda\right)^{\frac{1}{\lambda}}, 0\right) &, \text{ если }\lambda\in(0, +\infty)\\
      T_d(\mu_a, \mu_b) &, \text{ если }\lambda=0\\
      T_m(\mu_a, \mu_b) &, \text{ если }\lambda=\infty\\
   \end{cases}
$$

Проверка свойств:
\begin{enumerate}
    \item $$
      T(0, 0) = 
      \begin{cases}
         max\left(1-2^{\frac{1}{\lambda}}, 0\right) &, \text{ если }\lambda\in(0, +\infty)\\
         T_d(0, 0) &, \text{ если }\lambda=0\\
         T_m(0, 0) &, \text{ если }\lambda=\infty\\ 
      \end{cases} = 0;
      $$
      $$
      T(\mu_a, 1) = 
      \begin{cases}
         max\left(\mu_a, 0\right) &, \text{ если }\lambda\in(0, +\infty)\\
         T_d(\mu_a, 1) &, \text{ если }\lambda=0\\
         T_m(\mu_a, 1) &, \text{ если }\lambda=\infty\\ 
      \end{cases} = \mu_a;
      $$
  \item Коммутативность и монотонность(из свойств нормы и максимума) очевидна;
  \item Для ассоциативности проверим только 1-ый случай, остальные верны из определения t-нормы.
  Более того, для проверки ассоциативности \textbf{достаточно проверить следующее равенство:}
      $$ 
      max(1-y, 0) = 1-min(y, 1)
      $$
      1. $1-y<0 => y>1 => max(1-y, 0) = 0 = 1-min(y, 1)$
      2. $1-y>0 => 1>y => max(1-y, 0) = 1-y = 1-min(y, 1)$
      
      Тогда 
      $$
      max\left(1-\left((1-\mu_a)^\lambda+(1-\mu_b)^\lambda\right)^{\frac{1}{\lambda}},
      0\right) = 
      $$
      $$
      1-min\left(((1-\mu_a)^\lambda+(1-\mu_b)^\lambda)^{\frac{1}{\lambda}},
      1\right) 
      $$
      Отсюда следует ассоциативность:
      $$
      (1-\mu_a)^\lambda+(1-1+min\left(((1-\mu_b)^\lambda+(1-\mu_c)^\lambda)^{\frac{1}{\lambda}},
      1\right)^\lambda =
      $$
      $$
      (1-\mu_a)^\lambda + min\left((1-\mu_b)^\lambda+(1-\mu_c)^\lambda, 1\right)^\lambda
      $$
      (ассоциативность внешнего максимума показывается аналогично)
  \item Монтонность следует из п.3 и монотонности минимума
\end{enumerate}

\subsubsection{t-конорма}

\begin{equation*}
\perp(\mu_a, \mu_b) =
\begin{cases}
   min\left((\mu_a^\lambda+\mu_b^\lambda)^{\frac{1}{\lambda}}, 1\right) &, \text{ если }\lambda\in(0, +\infty)\\
   \perp_d(\mu_a, \mu_b) &, \text{ если }\lambda=0\\
   \perp_m(\mu_a, \mu_b) &, \text{ если }\lambda=\infty\\
\end{cases}
\end{equation*}

\textbf{Проверка свойств(только для первого случая, для конорм эти проверки уже выполнены):}
\begin{enumerate}
    \item $$
      \perp(1, 1) = min(2^\frac{1}{\lambda}, 1) = 1
      \newline
      \perp(\mu_a, 0) = min(\mu_a, 1) = \mu_a
      $$
  \item Коммутативность, ассоциативность и монотонность следуют из аналогичных рассуждений для
      t-нормы. 
\end{enumerate}

\end{document}
